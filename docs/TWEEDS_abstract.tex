\documentclass[11pt,english]{article}
\usepackage[T1]{fontenc}
\usepackage{babel}

\title{Large-Scale Marine Protected Areas displace fishing
effort and induce behavioral changes in fishing vessels}

\author{
  Juan CarlosVillase\~{n}or-Derbez\footnote{Bren School of Environmental Science \& Management, University of California Santa Barbara, Santa Barbara, CA} \\
  \texttt{juancarlos@ucsb.edu}
  \and
 John Lynham\footnote{Department of Economics, University of Hawaii at Manoa, Honolulu, HI}\\
  \texttt{lynham@hawaii.edu }
}

\date{}

\begin{document}
\maketitle
\begin{abstract}
Large-scale Marine Protected Areas (LSMPAs) have seen a significant
increase over the last years. Fishing effort is effectively eliminated
within these protected areas upon implementation. The benefits of
reducing effort have been largely studied, but little empirical works
evaluate how vessels react and redistribute after an MPA is created. The
economic and ecological implications of displacing fishing effort are
not yet fully understood. We use identification of fishing activity via
Automatic Identification Systems (AIS) and causal inference techniques
to provide the first analysis of behavioral changes and spatial
redistribution of tuna purse seiners due to the implementation of a
Large Scale Marine Protected Area in the Pacific Ocean. Our work
provides three main findings: 1) aggregate fishing effort remains
relatively unaffected; 2) vessels that fished inside the protected area
redistribute to adjacent waters; and 3) we observe a crowding effect for
the first months after implementation. Our results not only provide an
impact evaluation of the effect of LSMPAs on fishing activity, but
provide insights into vessel redistribution dynamics, which may have
ecological and economic implications for marine conservation and fisheries
management. As countries continue to implement LSMPAs as a way to
reach the stated 10\% target of ocean protection, managers should
consider how fishing effort will change in space and through time to
ensure that fishing effort is not just displaced elsewhere, leading to
overfishing in adjacent waters. While LSMPAs can provide a wide
range of benefits, their implementation must be accompained with
traditional fisheries management to maximize their effectiveness.
\end{abstract}

\end{document}
